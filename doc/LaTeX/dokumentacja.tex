\documentclass[pdflatex,11pt]{aghdpl}

\usepackage[polish]{babel}
\usepackage[utf8]{inputenc}
\usepackage[T1]{fontenc}

% dodatkowe pakiety
\usepackage{enumerate}
\usepackage{pdfpages}
\usepackage{afterpage}
\usepackage{pdflscape}
%\usepackage{rotating}

\graphicspath{{./img/}}
%\usepackage{subfigure} %kilka obrazkow w ramach jednego figure

%------------------------- STRONA TYTUŁOWA -------------------------

\author{Marta~Drabarczyk, Krzysztof~Kutt, Michał~Nowak}
\titlePL{Księgarnia internetowa \textbf{WhaToRead}}
\thesistypePL{Wzorce Projektowe}
\date{2012}
\departmentPL{Katedra Automatyki}
\facultyPL{Wydział Elektrotechniki, Automatyki, Informatyki i Elektroniki}
\setlength{\cftsecnumwidth}{10mm}

%------------------------- DOKUMENT START -------------------------

\begin{document}

\titlepages

\tableofcontents
\clearpage

%------------------------- OGÓLNY OPIS SYSTEMU -------------------------

\chapter{Ogólny opis systemu}

\section{Cel systemu}

Celem projektu było stworzenie prostego systemu informatycznego obsługującego księgarnię internetową z możliwościami zarządzania książkami i kategoriami oraz składania zamówień. W projekcie wykorzystano wzorce projektowe opisane w dalszej części dokumentacji.

%-------------------------

\section{Udziałowcy}

Właścicielami systemu będą firmy prowadzące księgarnie internetowe, zarówno mniejsze jak i większe, gdyż nasz system łatwo skaluje się do dowolnej ilości użytkowników i książek.

Grupę użytkowników wewnętrznych stanowią pracownicy prowadzący bieżącą działalność księgarni: zarządzają aktualną listą książek i kategorii oraz realizują zamówienie składane przez użytkowników.

Użytkownikami zewnętrznymi systemu są klienci księgarni internetowej.

%-------------------------

\section{Granice systemu}

Jedyną granicę systemu stanowi strona internetowa, zapewniająca dostęp do wszystkich funkcjonalności systemu.

%-------------------------

\section{Lista możliwości}

Dla wszystkich:
\begin{enumerate}
\item Rejestracja
\item Logowanie
\item Przeglądanie książek
\item Wyszukiwanie książek
\item Ocenianie książek
\end{enumerate}

Dla klientów:
\begin{enumerate}
\item Przeglądanie i zmiana danych
\item Składanie zamówień
\end{enumerate}

Dla pracowników:
\begin{enumerate}
\item Dodanie/modyfikacja/usunięcie kategorii
\item Dodanie/modyfikacja/usunięcie książki
\item Przeglądanie i zmiana statusów zamówień złożonych przez użytkowników
\end{enumerate}


%------------------------- ANALIZA DZIEDZINY -------------------------

\chapter{Analiza dziedziny}

\section{Baza danych}

\subsection{Produkty}
\begin{itemize}
\item ID produktu
\item Nazwa
\item Opis
\item Cena
\item ID Kategorii
\item Zdjęcie
\item Ilość ocen
\item Suma ocen
\item ilość produktów (na stanie)
\end{itemize}

\subsection{Kategorie produktów}
\begin{itemize}
\item ID
\item Nazwa
\item Opis
\end{itemize}

\subsection{Komentarze}
\begin{itemize}
\item ID produktu
\item ID użytkownika
\item Komentarz
\end{itemize}

\subsection{Zamówienia}
\begin{itemize}
\item ID zamówienia
\item ID klienta
\item dane klienta
\item lista ID produktów
\item Forma dostawy
\item Forma płatności
\item Stan zamówienia
\end{itemize}

\subsection{Użytkownicy}
\begin{itemize}
\item ID użytkownika
\item Imię
\item Nazwisko
\item Adres
\item Telefon
\item Mail
\end{itemize}


%------------------------- SPECYFIKACJA WYMAGAŃ -------------------------

\chapter{Specyfikacja wymagań}

\section{Przypadki użycia}

Klient:
\begin{enumerate}
\item        Rejestracja
\item        Logowanie
\item        Edycja danych
\item        Przeglądanie katalogu książek
\item        Wyszukiwanie
\item        Przeglądanie koszyka zakupów
\item        Dodanie książki do koszyka
\item        Usunięcie książki z koszyka
\item        Skomentowanie książki
\item        Ocenienie książki
\item        Złożenie zamówienia
\end{enumerate}

Pracownik księgarni:
\begin{enumerate}
\item        Założenie konta pracownika księgarni
\item        Dodawanie/Usuwanie kategorii książek
\item        Dodawanie/Usuwanie książek
\item        Zmienianie statusu realizacji zamówienia
\end{enumerate}

%-------------------------

\section{Wybrane funkcjonalności}

\begin{enumerate}
\item Każdy produkt reprezentowany w systemie posiada swoją nazwę, opis, zdjęcie i cenę. Jest dostępna również możliwość oceny danego produktu przez klientów zarówno poprzez skalę punktową, jak i wpisanie komentarza.
\item Podział produktów na kategorie (np. Książki dla dzieci / Książki dla dorosłych / Podręczniki / Czasopisma), które można dowolnie usuwać, bądź dodawać w zależności od aktualnych potrzeb.
\item System posiada funkcjonalność koszyka, do którego klienci mogą dodawać wybrane produkty, celem późniejszego złożenia zamówienia. Koszyk jest przechowywany w ramach aktualnej sesji.
\item Możliwość zakładania kont w serwisie. Dzięki temu klient ma dostęp do historii swoich zamówień i nie musia wpisywać swoich danych przy każdym zamówieniu, gdyż są one pobierane z jego konta.
\item Po poprawnym złożeniu zamówienia, klient otrzymuje mail z informacjami o tym zamówieniu (m.in. numer identyfikacyjny), zaś samo zamówienie pojawia się na liście aktywnych zamówień, dostępnej dla pracowników księgarni.
\item Obsługiwane przez system możliwości płatności to: gotówka przy odbiorze i przelew tradycyjny.
\end{enumerate}


%------------------------- WZORCE PROJEKTOWE -------------------------

\chapter{Wzorce projektowe}

\section{SEC 1}

Klimek jest głupi

%-------------------------

\section{SEC 2}

Klimek jest głupi

%------------------------- IMPLEMENTACJA -------------------------

\chapter{Implementacja}

\section{SEC 1}

Klimek jest głupi

%-------------------------

\section{SEC 2}

Klimek jest głupi


\end{document}

